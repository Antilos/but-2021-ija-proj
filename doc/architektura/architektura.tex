\documentclass[10pt,a4paper]{article}
\usepackage[utf8]{inputenc}
\usepackage[T1]{fontenc}
\usepackage[english]{babel}
\usepackage{amsmath}
\usepackage{amsfonts}
\usepackage{amssymb}
\usepackage{graphicx}
\usepackage{rotating}

\author{Jakub Kocalka, xkocal00; Juraj Sloboda, xslobo07}
\title{IJA Projekt 2021}
\begin{document}
	\maketitle
	
	\begin{sidewaysfigure}[ht]
		\includegraphics[width=\textwidth]{ija-diagram-division.png}
	\end{sidewaysfigure}

%	\includegraphics[width=\pagewidth]{ija-diagram-division.png}

	\section{Náčrt rozdelenia práce}
	\subsection{xkocal00}
	Košíky. Plnenie objednávok košíkmi. Košík vie prijať objednávku, nájsť najbližšií regál ktorý obsahuje potrebný tovar, nájsť cestu k tomuto regálu (A* pathfinding), a naplánovať si do kalendára udalosti potrebné k svojmu pohybu.
	
	Po príchode k regálu košík naberie tovar. Po zozbieraní potrebnáho tovaru si košík naplánuje cestu k výdajnému miestu.
	
	Košík v súčastnosti nepredpokladá že v obchode nie je dostatočné množstvo tovaru na splnenie objednávky, ani nepočíta s kapacitou košíku.
	
	\subsection{xslobo07} 
	Načítanie dát zo súboru (velkosť obchodu, typy tovaru, umiestnenia a mená regálov, tovar na regáloch, objednávky). Diskrétna mapa obchodu, zložená so štvorcov. Regále s tovarom, naskladnenie a vyskladnenie tovaru medzi košíkmi a regálmi.
	
\end{document}